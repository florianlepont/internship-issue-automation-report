\chapter{Conclusion}
\label{Conclusion}
\thispagestyle{fancy}

J'ai eu la chance de pouvoir réaliser mon stage de fin d'études au sein d'Aldebaran sous la tutelle de Emmanuel Nalepa, chef de l'équipe Qualification Hardware Pepper. Cette opportunité m'a permis de bénéficier d'une première expérience dans le monde Machine Learning, de la qualification et d'acquérir un ensemble de techniques indispensables à leurs pratique. Il s'agissait ici d'automatiser le processus d'investigation permettant de déterminer la cause ayant provoqué l'apparition d'une erreur durant le Filtering Test. Ce test vise à stresser l'ensemble des parties mécaniques en fin de chaîne de production, afin de révéler un maximum d'erreurs avant d'envoyer le produit chez le client.
\newline
Dans un premier temps, je me suis familiarisé avec les théories mathématiques qui fondent le Machine Learning. J'ai mis en pratique plusieurs des différentes méthodes algorithmiques afin d'en étudier leurs fonctionnement, leurs champs d'applications et les possibilités que celles-ci offraient au regard de notre problématique. J'ai ensuite comparé les différentes possibilités et émis un choix sur l'une d'entre elles : on effectuera de l'apprentissage automatique supervisé en s'appuyant sur l'algorithme \emph{Support Vector Machine} (SVM)
\newline
Dans un second temps, j'ai mis en place l'architecture fonctionnelle du système d'automatisation des investigations, en prenant en compte les contraintes imposées par les outils déjà mis en place, comme par exemple la structure des données générées lors de l'apparition d'une erreur durant le Filtering Test. 
\newline