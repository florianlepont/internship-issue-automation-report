\begin{abstract}
\label{abstract}
Mon stage ingénieur, réalisé dans le cadre de ma cinquième année de formation à l'École Nationale d'Ingénieurs de Brest (ENIB), s’est déroulé au sein du département "Qualification Hadware Pepper (QWP) de l'entreprise Aldebaran. La société Parisienne s'est fait connaitre dans le monde des nouvelles technologies et de la robotique humanoïde grâce au développement de son premier produit, "Nao". A l'origine, le robot se prédestine à l'univers de la recherche et aux universités. La société cherche aujourd'hui à conquérir de nouveaux marchés en offrant des produits et des services au monde de l'entreprise et aux particuliers. Cela se traduit notamment par le développement d'un tout nouveau produit: "Pepper". 

Cette extension du marché s'accompagne d'une montée en puissance de la fabrication de robots "Pepper". Cela induit le développement d'une nouvelle génération d'outils de production et de post-production. Un des dispositif mis en place est le "Filtering Test": à la fin de la chaine de production, les robots sont soumis à une série de tests qui visent à mettre à l'épreuve leurs différentes parties mécaniques et électroniques. Lorsqu'une erreur est détectée, les différentes données du robot son enregistrées (e.g. température des fusibles, valeurs de l'accéléromètre, etc.). Afin de déterminer qu'elles sont les causes qui ont entrainé l'apparition de l'anomalie sur le robot, chaque donnée est étudiée minutieusement et des hypothèses sont émises. Cette tâche dite d'investigation peu s'avérer laborieuse, il y'a donc un souhait fort d'automatiser ce processus.

Le but de  ma présence au sein d'Aldebaran est donc de répondre à ce besoin. En s'appuyant sur l'utilisation de méthodes d'apprentissage automatiques (en anglais, on parle de Machine Learning), j'ai donc mis au point un algorithme capable de déterminer automatiquement (après une phase d'apprentissage) les causes ayant entrainées l'apparition d'anomalies sur Pepper.  La mise au point de cet outil a été réalisé en trois temps:
\begin{enumerate}
	\item auto-formation à l'apprentissage automatique et de maitrise des outils.
	\item conception et développement de l'algorithme
	\item industrialisation du produit, c'est à dire en simplifier l'utilisation et le robustifier.
\end{enumerate}
 

Le rapport de stage ci-joint présente les différentes recherches effectuées ainsi que les travaux réalisés pour répondre au mieux à la problématique initiale.\end{abstract}
