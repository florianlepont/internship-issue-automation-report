\chapter{Introduction}
\label{Introduction}
\thispagestyle{fancy}

\section{Contexte}
\label{Introduction:Contexte}

\subsection{Filtering Test}
\label{Introduction:Contexte:Filtering Test}
L'extension du marché visé par Aldebaran pour Pepper s'accompagne d'une montée en puissance de la production. Afin de l'accompagner, des outils de vérification des produits en fin de ligne de production sont mis en place. Parmi celui-ci le "Filtering Test", ce test de six heures vise à stresser l'ensemble des parties mécaniques du robot afin de faire ressortir d'éventuels problèmes ou erreurs sur le robot. Si une erreur apparait lors du déroulement du test, différentes données du robots robot sont enregistrées 


 qui nécessite d'être accompagné d'une nouvelle génération d'outils de production et post-production. Un des dispositif mis en place est le "Filtering Test": A la fin de la chaine de production, les robots sont soumis à plusieurs tests visant à mettre à l'épreuve les différentes parties mécaniques et électroniques. Lorsqu'une erreur est détectée,  les différentes données du robot son enregistrées (e.g. température des fusibles, valeurs de l'accéléromètre, etc.). Afin de déterminer qu'elles sont les causes qui ont entrainé l'apparition de l'anomalie sur le robot, chaque donnée est étudiée minutieusement et des hypothèses sont émises. Cette tâche dite d'investigation peu s'avérer laborieuse, le souhait d'automatiser ce processus est donc important.