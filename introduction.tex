\chapter{Introduction}
\label{Introduction}
\thispagestyle{fancy}

\section{Expression du besoin}
\label{Introduction:Expression du besoin}
L'extension du marché visée par Aldebaran pour Pepper s'accompagne d'une montée en puissance de la production. Afin de l'accompagner, des outils de vérification des produits en fin de ligne de production sont mis en place. Parmi celui-ci le "Filtering Test", ce test de six heures vise à stresser l'ensemble des parties mécaniques du robot afin de faire ressortir d'éventuels problèmes ou erreurs sur le robot. Si une anomalie apparait lors du déroulement du test, différentes données relatives à l'état des systèmes mécaniques et électroniques du robot sont enregistrées dans un fichier journal (e.g. température des fusibles, valeur des accéléromètres, etc.). Afin d'identifier les causes de l'apparitions de problèmes sur le robot, un certain nombre d'hypothèses sont émises à partir des données en sortie du système. 

\paragraph{Exemple d'analyse d'un fichier}
\label{Introduction:Expression du besoin:Exemple d'analyse d'un fichier}
\begin{enumerate}
	\item Lors du déroulement du test, le robot tombe à t = 16972 secondes.
	\item En atteste les valeurs retournées par l'accéléromètre, selon l'axe $Z$
	\item On analyse les données liées aux systèmes mécanique et électroniques du robot pouvant avoir une relation directe ou indirecte avec sa chute.  
	Lorsque l'on étudie la vitesse de rotation du moteur de la hanche, on remarque qu'à environs de  t = 16970 secondes (c'est-à-dire 2 secondes avant la chute du robot),  l'information fournie par le senseur ne suit plus la commande  envoyée au moteur.
\end{enumerate}

\subsection{Présentation du produit}
\label{Introduction:Expression du besoin:Présentation du produit}

\section{Solution proposée}
De part la quantité d'informations à analyser, cette tâche peut rapidement devenir rébarbative, d'où le souhait d'automatiser ce processus d'investigation. De part la variabilité des types de données, on ne peut réduire le nombre d'information à analyser en recherchant des caractéristiques communes (e.g. moyenne, écart type, etc.). On se propose donc d'utiliser des méthodes algorithmes plus poussées  et notamment des méthodes dites  d'apprentissages automatiques (plus connues sous le terme anglais de Machine Learning). 