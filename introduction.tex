\chapter{Introduction}
\label{Introduction}
\thispagestyle{fancy}

\section{Contexte}
\label{Introduction:Contexte}

 L'entreprise s'est fait connaitre dans le monde des nouvelles technologie et de la robotique humanoïde grâce au développement du robot "Nao", à destination des université et laboratoires de recherche. La société cherche aujourd'hui à étendre son marché aux entreprises et aux particuliers avec le développement d'un tout nouveau produit: "Pepper". 
 
 Cette extension du marché s'accompagne d'une montée en puissance de la production des robots "Pepper" qui nécessite d'être accompagné d'une nouvelle génération d'outils de production et post-production. Un des dispositif mis en place est le "Filtering Test": A la fin de la chaine de production, les robots sont soumis à plusieurs tests visant à mettre à l'épreuve les différentes parties mécaniques et électroniques. Lorsqu'une erreur est détectée,  les différentes données du robot son enregistrées (e.g. température des fusibles, valeurs de l'accéléromètre, etc.). Afin de déterminer qu'elles sont les causes qui ont entrainé l'apparition de l'anomalie sur le robot, chaque donnée est étudiée minutieusement et des hypothèses sont émises. Cette tâche dite d'investigation peu s'avérer laborieuse, le souhait d'automatiser ce processus est donc important.