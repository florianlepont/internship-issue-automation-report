\chapter{Introduction}
\label{Introduction}
\thispagestyle{fancy}

\section{Contexte}
\label{Introduction:Contexte}

\subsection{Filtering Test}
\label{Introduction:Contexte:Filtering Test}
L'extension du marché visé par Aldebaran pour Pepper s'accompagne d'une montée en puissance de la production. Afin de l'accompagner, des outils de vérification des produits en fin de ligne de production sont mis en place. Parmi celui-ci le "Filtering Test", ce test de six heures vise à stresser l'ensemble des parties mécaniques du robot afin de faire ressortir d'éventuels problèmes ou erreurs sur le robot. Si une erreur apparait lors du déroulement du test, différentes données relatives à l'état des systèmes mécaniques et électroniques du robot sont enregistrées dans un fichier journal (e.g. température des fusibles, valeur des accéléromètres, etc.). Afin d'identifier les causes de l'apparitions de problèmes sur le robot, un certain nombre d'hypothèses sont émises à partir des données en sortie du système. 

\paragraph{Exemple d'analyse d'un fichier}
\begin{enumerate}
	\item Lors du déroulement du test, le robot tombe.
	\item En atteste les valeurs retournées par l'accéléromètre
\end{enumerate}


De part la quantité d'informations à analyser, cette tâche peut rapidement devenir rébarbative, d'où le souhaite d'automatiser ce processus d'investigation.
