\chapter{ Le  Machine Learning}
\label{Le Machine Learning}
\thispagestyle{fancy}

\section{Généralités sur le Machine Learning}
\label{Le Machine Learning: Généralités sur le Machine Learning}
Le machine learning (traduire par apprentissage automatique) constitue une branche de l'intelligence artificielle. Son statut de subdivision ne renseigne cependant en rien sur l'étendu des notions contenues dans cette matière scientifique. Son champ d'étude est vaste et en perpétuelle évolution. Les solutions offertes par cette discipline permettent d'étudier toute sortes de données et d'automatiser une multitude de systèmes. L'apprentissage automatique rencontre un succès croissant qui est corrélé avec l'essor des nouvelles technologies et l'automatisation de l'analyse de volumes conséquents de données utilisateurs (Big Data). Les applications sont donc multiples, en voici quelques exemples:  

\begin{itemize}
	\item Algorithmes des moteurs de recherches (Google Deep Dream, Google TensorFlow)
	\item Analyse boursière
	\item Analyse de rapports d'erreurs
	\item Reconnaissance vocale, biométrie, reconnaissance d'écriture
	\item Robotique (vision, mouvements, prise de décision, etc.)
	\item Neurosciences 
\end{itemize}

\subsection{Définition et principe général}
\label{Le Machine Learning: Généralités sur le Machine Learning: Définition et principe général}
Le champ d'étude et d'application du Machine Learning étant immense, on propose de définir cette notion pour l'adapter à la résolution de notre problématique, c'est-à-dire d'automatiser l'analyse d'incidents révélés lors du filtering test.
On offre ici deux définitions de l'apprentissage automatique: une première dite "High Level" qui le caractérise de manière générale , une seconde qui reflète sa dimension algorithmique. 

\begin{description}
	\item[High Level]: Le Machine Learning permet à un système d'évoluer grâce à un processus d'apprentissage et ainsi de remplir des tâches qu'il est difficile, voir impossible, de remplir par d'autres moyens algorithmiques plus classiques. 
	\item[Mathématique] Le Machine Learning fourni les outils pour prédire une/des donnée(s) de sortie Y à partir des données d'entrée X via un processus d'apprentissage. 
\end{description}
 
 De nombreuses autres définitions existent pour définir ce qu'est le Machine Learning, mais elles ne correspondent pas à la dimension recherchée dans le cadre de ce projet.
 
Au regard des deux définitions stipulées ci-dessus, on peut définir le principe de base de l'apprentissage automatique sous la forme d'un schéma bloc.
% Ajouter le schéma bloc high level du machine learning icitte 

L'apprentissage automatique peut être vu dans sa globalité comme un processus linéaire: 
\begin{enumerate}
	\item  On a en entrée du système une ou des donnée(s) brutes.  
	\item cette donnée est traitée et analysée par l'algorithme.
	\item En sortie, une décision est prise quant à la tâche demandée. 
\end{enumerate}

De plus, une tâche  dite d'apprentissage précède ce processus: 
\begin{enumerate}
	\item  Un ensemble de données sont présentées à l'algorithme, .  
	\item En sortie, une décision est prise quant à la tâche demandée. 
\end{enumerate}

Afin de présenter de manière plus concrète le 
\subsection{Les données}
\label{Le Machine Learning: Généralités sur le Machine Learning: Les données}

\subsection{La décision}
\label{Le Machine Learning: Généralités sur le Machine Learning: La décision}

\subsection{Le modèle}
\label{Le Machine Learning: Généralités sur le Machine Learning: Le modèle}


\section{Les différents algorithmes}
\label{ILe Machine Learning: Les différents algorithmes}

\subsection{La regression logistique}
\label{ILe Machine Learning: Les différents algorithmes: La regression logistique}

\subsection{Les réseaux neuronaux}
\label{ILe Machine Learning: Les différents algorithmes: Les réseaux neuronaux}

\subsection{SVM - Support Vector Machine-}
\label{ILe Machine Learning: Les différents algorithmes: SVM}

\subsection{Comparaison des algorithmes}
\label{ILe Machine Learning: Les différents algorithmes: Comparaison des algorithmes}