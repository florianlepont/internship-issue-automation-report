\chapter{ Le  Machine Learning}
\label{Le Machine Learning}
\thispagestyle{fancy}

\section{Généralités sur le Machine Learning}
\label{Le Machine Learning: Généralités sur le Machine Learning}
Le machine learning (traduire par apprentissage automatique) est une ramification de l'intelligence artificielle. Cependant, son statut de subdivision n'informe en rien sur l'étendu des notions contenues dans cette matière scientifique. Son champ d'étude est vaste et en perpétuelle évolution. Les solutions offertes par cette discipline permettent d'étudier toute sortes de données et d'automatiser une multitude de systèmes. L'apprentissage automatique rencontre un succès croissant qui est corrélé avec l'essor des nouvelles technologies et l'automatisation de l'analyse de volumes conséquents de données utilisateurs (Big Data). Les applications sont donc multiples. En voici quelques exemples:  

\begin{itemize}
	\item Algorithmes des moteurs de recherches (Google Deep Dream, Google TensorFlow)
	\item Analyse boursière
	\item Analyse de rapports d'erreurs
	\item Reconnaissance vocale, biométrie, reconnaissance d'écriture
	\item Robotique (vision, mouvements, prise de décision, etc.)
	\item Neurosciences 
\end{itemize}

\subsection{Définition et principe général du Machine Learning}
\label{Le Machine Learning: Généralités sur le Machine Learning: Définition et principe général}
Le champ d'étude et d'application du Machine Learning étant immense, on propose de redéfinir cette notion en l'adaptant à la résolution de notre problématique (i.e. automatiser l'analyse d'incidents révélés lors du filtering test).
On offre ici deux définitions de l'apprentissage automatique: une première dite "High Level" qui le caractérise de manière générale et une seconde qui reflète sa dimension algorithmique. 

\begin{description}
	\item[High Level]: Le Machine Learning permet à un système d'évoluer grâce à un processus d'apprentissage et ainsi de remplir des tâches qu'il est difficile, voir impossible, de remplir par d'autres moyens algorithmiques plus classiques. 
	\item[Mathématique] Le Machine Learning fourni les outils pour prédire une/des donnée(s) de sortie Y à partir des données d'entrée X via un processus d'apprentissage. 
\end{description}
 
 De nombreuses autres définitions existent, mais elles ne correspondent pas à la dimension recherchée dans le cadre de ce projet.
 
Au regard des deux définitions stipulées ci-dessus, on peut définir le principe de base de l'apprentissage automatique sous la forme d'un schéma bloc.
% Ajouter le schéma bloc high level du machine learning icitte 

L'apprentissage automatique peut donc être vu dans sa globalité comme un processus composé de deux étapes successives : 
\begin{enumerate}
		\item [Apprentissage]
		 \begin{enumerate}
			\item  Un ensemble de données est présenté au système (X\_train).
			\item A partir de ces informations, le système (f) apprend - s'entraîne- afin d'être par la suite en capacité de prendre décision vis à vis de la tâche qui lui sera demandée. 
		\end{enumerate}
		
		\item [Prise de décision] 
		\begin{enumerate}
			\item  On a en entrée du système une ou des donnée(s) brutes (X).  
			\item Cette donnée est traitée et analysée par le système.
			\item En sortie, une décision est prise quant à la tâche demandée (Y). 
		\end{enumerate}
\end{enumerate}

\subsubsection{Un exemple concret}
\label{Le Machine Learning: Généralités sur le Machine Learning: Définition et principe général:un exemple concret}
Afin de présenter de manière plus concrète le processus fonctionnel haut niveau d'un algorithme d'apprentissage, on présente l'exemple suivant:

\textit{On cherche à déterminer à quelle période de l'année nous nous trouvons (i.e. printemps, été, automne ou hiver) à partir de l'humidité, la température et la pression atmosphérique d'aujourd'hui. \\
	La première étape de notre processus sera donc d'entrainer notre système afin que celui-ci soit en mesure de prendre une décision vis à vis des données qu'on lui présentera en entrée (i.e. l'humidité, la température et la pression atmosphérique d'aujourd'hui). \\
	Une fois le système entrainé, on attend que celui-ci ai ce type de comportement : \\
	On présente en entrée de mon système une température de -2 degrés, une pression atmosphérique de 1030hPa et un taux d'humidité de 81\%. La réponse attendue en sortie du système est: hiver}

Pour caractériser et  désigner plus précisément les différents éléments de notre système, on présente ci-dessous le champs lexical utilisé dans le domaine du Machine Learning:

\subsubsection{Lexique} 
\begin{description}
	\item [les features] Le type de données présenté en entrée. \\
	\textit{La température, la pression atmosphérique et l'humidité.}
	\item [échantillons ou exemples] Les données permettant d'entrainer le système (x\_train). \\
	\textit{De nombreux échantillons de température, pression atmosphériques et humidité pris à différents périodes de l'année, sur plusieurs années.}
	\item [le modèle d'apprentissage] Le cœur du système décisionnel (f).
	\item [La décision] La sortie ou réponse du système (Y) \\
	\textit{Printemps, été, automne ou hiver.}
\end{description}

\subsection{Les exemples}
\label{Le Machine Learning: Généralités sur le Machine Learning: Les données}
Les exemples correspondent aux données utilisées pour entraîner mon algorithme d'apprentissage. On parle également d'échantillons. Celles-ci sont regroupées en "features" (terme anglais, traduire par caractéristiques). 
Pour reprendre l'exemple cité précédemment \ref{Le Machine Learning: Généralités sur le Machine Learning: Définition et principe général:un exemple concret}, nos données sont regroupées en 3 features: la température, l'humidité et la pression atmosphérique. On données sont donc structurées de la manière suivante: 

\begin{equation}
\begin{blockarray}{cccc}
& température (\degres C) & humidite (\%) & pression(HPa) \\
\begin{block}{c(ccc)}
Exemple_1 & -10 & 85 & 1023 \\
Exemple_2 & 15 & 80 & 1020 \\
Exemple_3 & 23 & 65 & 1015 \\
... & ... & ... & ... \\
Exemple_n & 10 & 81 &  1032 \\
\end{block}
\end{blockarray}
\end{equation}

\subsubsection{Différents types de données}
Il existe deux types de données: les données labellisées et non labellisées.
\begin{itemize}
	\item Les données labellisées correspondent à des exemples corrélés à une sortie - un label - connue.
	\item Les données non labellisées ne sont quant à elles pas associées à une sortie. 
\end{itemize}

Pour reprendre l'exemple précédent \ref{Le Machine Learning: Généralités sur le Machine Learning: Définition et principe général:un exemple concret}, on a les jeux de données suivants: 

\textbf{Données labellisées} 
\begin{equation}
\begin{blockarray}{ccccc}
& température (\degres C) & humidite (\%) & pression(HPa) \\
\begin{block}{c(ccc)c}
Exemple_1 & -10 & 85 & 1023 & hiver\\
Exemple_2 & 15 & 80 & 1020 & automne\\
Exemple_3 & 23 & 65 & 1015 & ete \\
... & ... & ... & ... \\
Exemple_n & 10 & 81 &  1032 & printemps\\
\end{block}
\end{blockarray}
\end{equation}
On connait la sortie qui correspond aux données d'entrée, i.e. que on sait à quelle période de l'année les échantillons ont été prélevé.
 
\textbf{Données non labellisées} 
\begin{equation}
\begin{blockarray}{ccccc}
& température (\degres C) & humidite (\%) & pression(HPa) \\
\begin{block}{c(ccc)c}
Exemple_1 & -10 & 85 & 1023 & ??\\
Exemple_2 & 15 & 80 & 1020 & ??\\
Exemple_3 & 23 & 65 & 1015 & ?? \\
... & ... & ... & ... \\
Exemple_n & 10 & 81 &  1032 & ??\\
\end{block}
\end{blockarray}
\end{equation}
On ne connait pas la sortie qui correspond aux données d'entrée, i.e. que on \emph{ne} sait \emph{pas} à quelle période de l'année les échantillons ont été prélevé. 

\subsection{La décision}
\label{Le Machine Learning: Généralités sur le Machine Learning: La décision}
La sortie de notre système peut être également nommé décision. Toujours selon notre exemple \ref{Le Machine Learning: Généralités sur le Machine Learning: Définition et principe général:un exemple concret}, cela correspond au choix fait par l'algorithme entre les différentes saisons: printemps, été, automne et hiver.  

\subsubsection{Différents types de sorties}
Il existe différents types de sortie: les sorties continues et discrètes.

\begin{description}
	\item [Les sorties continues] peuvent prendre n'importe quelle valeur. \\
	 $y \in R$ \\
	 Déterminer l'évolution de la température en fonction des échantillons enregistrés les mois précédents correspond à une sortie continue.
	\item [Les sorties discrètes] ne peuvent prendre que des valeurs prédéterminées. \\
	 $y \in {1, 2, 3, ...,C}$ \\
	 L'exemple \ref{Le Machine Learning: Généralités sur le Machine Learning: Définition et principe général:un exemple concret} a une sortie discrète. En effet, la sortie ne peut prendre que des valeurs prédéterminées: printemps, été, automne et hiver.
\end{description}



\subsection{Le modèle}
\label{Le Machine Learning: Généralités sur le Machine Learning: Le modèle}
Il existe différents types d'apprentissages. Le choix d'un modèle en particulier est influencé par le type d'exemples que l'on a en entrée du système et du type de décision que l'on souhaite obtenir en sortie. Nous nous intéresserons dans le cadre de notre études à deux apprentissages: l'apprentissage supervisé et l'apprentissage non supervisé.

\subsubsection{Comparaison de l'apprentissage supervisé et non supervisé} 
\label{Le Machine Learning: Généralités sur le Machine Learning: Le modèle: Apprentissage supervisé}
L'apprentissage supervisé nécessite d'avoir des données labellisées en entrée, i.e. que l'on connait le type de décision que l'on aura en sortie du système en fonction des exemples en entrée: il y'a une corrélation entre la sortie et l'entrée. C'est cette notion qui s'exprime au travers du terme \emph{supervisé}. 
L'apprentissage non supervisé s'appuie quant à lui sur l'utilisation d'une base de donnée non labellisée pour son apprentissage, i.e qu'on ne connait pas le type de décision associé aux exemples en entrée. Afin de matérialiser les différences entres les deux méthodes et les applications possibles pour chacune d'elles, on propose deux exemples \ref {tab: Comparaison des différentes méthodes d'apprentissage}.

\begin{table}[h]
	\begin{tabular}{ | p {2.5cm} | p {6cm} | p {6cm} |}
	\hline
	 & apprentissage supervisé & apprentissage non supervisé \\
	\hline
	\begin{tabular}{c} exemple n\degres1:\\apprendre aux \\ humains \end{tabular}  &
	 Une institutrice souhaite apprendre à ses élèves à différencier un chat d'un chien: c'est la décision qu'on attend d'eux. Pour ce faire, l'éducatrice leur montre différentes photographies de chiens et de chats: ce sont les exemples utilisés pour l'apprentissage. Ces exemples peuvent être ségementés en différentes caractéristiques, comme la taille de l'animal, sa couleur, la longueur du poil, etc: il s'agit des features. Lorsque que l'institutrice leur présente les différentes images, elle stipule clairement si il s'agit d'un chien ou d'un chat: il y'a donc une corrélation entre l'entrée et la sortie de l'apprentissage, il s'agit d'un apprentissage supervisé.&
	 On retrouve cette même institutrice donnant à ses élèves le même exercice à la différence que, contrairement à précédemment, lorsque qu'elle présente les différentes images, elle \emph{ne} stipule \emph{pas} la race de l'animal: il n'y a donc aucune corrélation entre l'entrée et la sortie de l'apprentissage, il s'agit donc d'un apprentissage non supervisé. Pour réussir cet exercice, les enfants devront donc regrouper les animaux grâce à leurs similitudes physiques, i.e. leurs features (e.g. taille de l'animal, sa couleur, longueur du poil, etc.). Les élèves ne connaitrons certes pas le nom des deux animaux, mais ils auront su les différencier. C'est la même approche qui est réalisé en apprentissage automatique non supervisé. \\
	\hline 
	\begin{tabular}{c} exemple n\degres2:\\la météo \ref*{Le Machine Learning: Généralités sur le Machine Learning: Définition et principe général:un exemple concret}\end{tabular} & On reprend l'exemple où l'on souhaite prendre une décision quant à la période de l'année à laquelle on se trouve actuellement,en fonction des features humidité, température et pression atmosphérique. Pour se faire, on prélève des échantillons à différentes périodes de l'année en notant à quelle saison ces données ont été prélevées: se sont donc des exemples labellisées, i.e. il y'a une corrélation entre les données et la sortie du système. Il s'agit donc d'un apprentissage supervisé. &
	On reprend le même problème mais cette fois-ci on ne note pas la saison à laquelle les échantillons ont été prélevé. Pour résoudre le problème, l'algorithme doit donc associer les données les plus similaires entre elles et ainsi créer des groupes homogènes d'informations qui correspondront aux 4 décisions possibles. \\
	\hline
	\end{tabular}
	\caption[Comparaison des différents modèles d'apprentissage]{Comparaison de l'apprentissage supervisé et non supervisé par des exemples}
	\label {tab: Comparaison des différentes méthodes d'apprentissage}
\end{table}


\subsubsection{Apprentissage non supervisé} 
\label{Le Machine Learning: Généralités sur le Machine Learning: Le modèle: Apprentissage non supervisé}

\section{Les différents algorithmes}
\label{ILe Machine Learning: Les différents algorithmes}

\subsection{La régression logistique}
\label{ILe Machine Learning: Les différents algorithmes: La regression logistique}

\subsection{Les réseaux neuronaux}
\label{ILe Machine Learning: Les différents algorithmes: Les réseaux neuronaux}

\subsection{SVM - Support Vector Machine-}
\label{ILe Machine Learning: Les différents algorithmes: SVM}

\subsection{Comparaison des algorithmes}
\label{ILe Machine Learning: Les différents algorithmes: Comparaison des algorithmes}