\chapter{ Le  Machine Learning}
\label{Le Machine Learning}
\thispagestyle{fancy}
Le machine learning (traduire par apprentissage automatique) constitue une branche de l'intelligence artificielle. Son statut de subdivision ne renseigne cependant en rien sur l'étendu des notions contenues dans cette matière scientifique. Son champ d'étude est vaste et en perpétuelle évolution. Les solutions offertes par cette discipline permettent d'étudier toute sortes de données et d'automatiser une multitude de systèmes. L'apprentissage automatique rencontre un succès croissant qui est corrélé avec l'essor des nouvelles technologies et l'automatisation de l'analyse de volumes conséquents de données utilisateurs (Big Data). Le Machine Learning est donc par exemple utilisé 

\section{Généralités sur le Machine Learning}
\label{Le Machine Learning: Généralités sur le Machine Learning}

\subsection{Définition}
\label{Le Machine Learning: Les différents algorithmes: Définition}

\subsection{Principes généraux}
\label{Le Machine Learning: Les différents algorithmes: Principes généraux}

\subsection{Les données}
\label{Le Machine Learning: Les différents algorithmes: Les données}

\subsection{La décision}
\label{Le Machine Learning: Les différents algorithmes: La décision}

\subsection{Le modèle}
\label{Le Machine Learning: Les différents algorithmes: Le modèle}


\section{Les différents algorithmes}
\label{ILe Machine Learning: Les différents algorithmes}

\subsection{Les différents algorithmes}
\label{ILe Machine Learning: Les différents algorithmes: La regression logistique}

\subsection{Les réseaux neuronaux}
\label{ILe Machine Learning: Les différents algorithmes: Les réseaux neuronaux}

\subsection{SVM - Support Vector Machine-}
\label{ILe Machine Learning: Les différents algorithmes: SVM}

\subsection{Comparaison des algorithmes}
\label{ILe Machine Learning: Les différents algorithmes: Comparaison des algorithmes}