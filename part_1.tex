\chapter{ Le  Machine Learning}
\label{Le Machine Learning}
\thispagestyle{fancy}

\section{Généralités sur le Machine Learning}
\label{Le Machine Learning: Généralités sur le Machine Learning}
Le machine learning (traduire par apprentissage automatique) est une ramification de l'intelligence artificielle. Cependant, son statut de subdivision n'informe en rien sur l'étendu des notions contenues dans cette matière scientifique. Son champ d'étude est vaste et en perpétuelle évolution. Les solutions offertes par cette discipline permettent d'étudier toute sortes de données et d'automatiser une multitude de systèmes. L'apprentissage automatique rencontre un succès croissant qui est corrélé avec l'essor des nouvelles technologies et l'automatisation de l'analyse de volumes conséquents de données utilisateurs (Big Data). Les applications sont donc multiples. En voici quelques exemples:  

\begin{itemize}
	\item Algorithmes des moteurs de recherches (Google Deep Dream, Google TensorFlow)
	\item Analyse boursière
	\item Analyse de rapports d'erreurs
	\item Reconnaissance vocale, biométrie, reconnaissance d'écriture
	\item Robotique (vision, mouvements, prise de décision, etc.)
	\item Neurosciences 
\end{itemize}

\subsection{Définition et principe général}
\label{Le Machine Learning: Généralités sur le Machine Learning: Définition et principe général}
Le champ d'étude et d'application du Machine Learning étant immense, on propose de redéfinir cette notion en l'adaptant à la résolution de notre problématique (i.e. automatiser l'analyse d'incidents révélés lors du filtering test).
On offre ici deux définitions de l'apprentissage automatique: une première dite "High Level" qui le caractérise de manière générale et une seconde qui reflète sa dimension algorithmique. 

\begin{description}
	\item[High Level]: Le Machine Learning permet à un système d'évoluer grâce à un processus d'apprentissage et ainsi de remplir des tâches qu'il est difficile, voir impossible, de remplir par d'autres moyens algorithmiques plus classiques. 
	\item[Mathématique] Le Machine Learning fourni les outils pour prédire une/des donnée(s) de sortie Y à partir des données d'entrée X via un processus d'apprentissage. 
\end{description}
 
 De nombreuses autres définitions existent, mais elles ne correspondent pas à la dimension recherchée dans le cadre de ce projet.
 
Au regard des deux définitions stipulées ci-dessus, on peut définir le principe de base de l'apprentissage automatique sous la forme d'un schéma bloc.
% Ajouter le schéma bloc high level du machine learning icitte 

L'apprentissage automatique peut donc être vu dans sa globalité comme un processus composé de deux étapes successives : 
\begin{enumerate}
		\item [Apprentissage]
		 \begin{enumerate}
			\item  Un ensemble de données est présenté au système (X\_train).
			\item A partir de ces informations, le système (f) apprend - s'entraîne- afin d'être par la suite en capacité de prendre décision vis à vis de la tâche qui lui sera demandée. 
		\end{enumerate}
		
		\item [Prise de décision] 
		\begin{enumerate}
			\item  On a en entrée du système une ou des donnée(s) brutes (X).  
			\item Cette donnée est traitée et analysée par le système.
			\item En sortie, une décision est prise quant à la tâche demandée (Y). 
		\end{enumerate}
\end{enumerate}

\paragraph{Un exemple concret}
Afin de présenter de manière plus concrète le processus fonctionnel haut niveau d'un algorithme d'apprentissage, on présente l'exemple suivant:

\textit{On cherche à déterminer à quelle période de l'année nous nous trouvons (i.e. printemps, été, automne ou hiver) à partir de l'humidité, la température et la pression atmosphérique d'aujourd'hui. \\
	La première étape de notre processus sera donc d'entrainer notre système afin que celui-ci soit en mesure de prendre une décision vis à vis des données qu'on lui présentera en entrée (i.e. l'humidité, la température et la pression atmosphérique d'aujourd'hui). \\
	Une fois le système entrainé, on attend que celui-ci ai ce type de comportement : \\
	Je présente en entrée de mon système une température de -2 degrés, une pression atmosphérique de 1030hPa et un taux d'humidité de 81\%. La réponse attendue en sortie du système est: Hiver}

Afin de caractériser et de désigner plus précisément les différents éléments de notre système, on présente ci-dessous le champs lexical utilisé dans le domaine du Machine Learning:

\paragraph{Lexique} 
\begin{description}
	\item [les features] Le type de données présenté en entrée. \\
	\textit{La température, la pression atmosphérique et l'humidité.}
	\item [échantillons ou exemples] Les données permettant d'entrainer le système (X\_train). \\
	\textit{De nombreux échantillons de température, pression atmosphériques et humidité pris à différents périodes de l'année, sur plusieurs années.}
	\item [le modèle d'apprentissage] Le cœur du système décisionnel (f).
	\item [La décision] La sortie ou réponse du système (Y) \\
	\textit{Printemps, été, automne ou hiver.}
\end{description}

\subsection{Les exemples}
\label{Le Machine Learning: Généralités sur le Machine Learning: Les données}
Les exemples utilisées pour entraîner mon algorithme d'apprentissage données brutes utilisées pour entraîner 
\subsection{La décision}
\label{Le Machine Learning: Généralités sur le Machine Learning: La décision}

\subsection{Le modèle}
\label{Le Machine Learning: Généralités sur le Machine Learning: Le modèle}


\section{Les différents algorithmes}
\label{ILe Machine Learning: Les différents algorithmes}

\subsection{La regression logistique}
\label{ILe Machine Learning: Les différents algorithmes: La regression logistique}

\subsection{Les réseaux neuronaux}
\label{ILe Machine Learning: Les différents algorithmes: Les réseaux neuronaux}

\subsection{SVM - Support Vector Machine-}
\label{ILe Machine Learning: Les différents algorithmes: SVM}

\subsection{Comparaison des algorithmes}
\label{ILe Machine Learning: Les différents algorithmes: Comparaison des algorithmes}