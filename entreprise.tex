\chapter*{Entreprise}
\label{Entreprise}
\thispagestyle{fancy}

\section{Histoire}
\label{Entreprise: histoire}
Aldebaran (anciennement Aldebaran Robotics) est une société Française de robotique humanoïde, fondée en 2005 par Bruno Maisonnier. 

\subsection{l'Aire Nao}
\label{Entreprise:Histoire:Nao}
Constituée au départ d'une équipe de douze collaborateurs, la toute jeune entreprise se fixe comme objectif de développer des robots humanoïdes et de les commercialiser au grand public, en tant que "nouvelle espèce bienveillante à l'égard des humains" . Après trois années de recherche et développement, la société dévoile en 2008 son tout premier robot: NAO. La participation du robot humanoïde à divers évènements internationaux (e.g. RoboCup, Exposition Universelle de Shanghai en 2010, etc.) participe à sa popularisation auprès des laboratoires de recherche, des universités et des développeurs. Une seconde génération de robot Nao apparait en 2011 (dit Nao Next Gen). L'entreprise dévoile durant la même période un nouveau projet en partenariat avec différents acteurs de la recherche,  qui vise à développer un véritable robot d'assistance à la personne, Roméo. 

\subsection{La famille Aldebaran}
\label{Entreprise: La famille Aldebaran}
Durant l'année 2012, Aldebaran est racheté par Softbank, société spécialisée dans le commerce électronique au Japon. Débute alors le développement d'un tout nous produit, le robot humanoïde Pepper. Dévoilé au grand public en 2014, il est dans un premier temps vendus au Japon auprès des entreprises. Les premiers clients à bénéficier de la présence de Pepper seront alors les magasins de téléphonie mobile de Softbank. Les ventes s'ouvrent dans un second temps au particuliers Japonais. La société poursuit à présent le développement de ses trois produits afin de les améliorer et de conquérir de nouveaux marchés (en Europe , en Chine et aux États-Unis)

\section{Les produits}
\label{Entreprise: Les produits}
Aldebaran commercialise à ce jours deux produits: Nao et Pepper
\subsection{Nao}
\label{Entreprise: Les produits: Nao}

\subsection{Pepper}
\label{Entreprise: Les produits: Pepper}

\subsection{Roméo}
\label{Entreprise: Les produits: Roméo}

