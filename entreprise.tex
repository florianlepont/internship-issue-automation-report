\chapter{Entreprise}
\label{Entreprise}
\thispagestyle{fancy}

\section{Histoire}
\label{Entreprise: histoire}
Aldebaran (anciennement Aldebaran Robotics) est une société Française de robotique humanoïde, fondée en 2005 par Bruno Maisonnier. 

\subsection{l'Aire Nao}
\label{Entreprise:Histoire:Nao}
Constituée au départ d'une équipe de douze collaborateurs, la toute jeune entreprise se fixe comme objectif de développer des robots humanoïdes et de les commercialiser au grand public, en tant que "nouvelle espèce bienveillante à l'égard des humains" . Après trois années de recherche et développement, la société dévoile en 2008 son tout premier robot: Nao. La participation du robot humanoïde à divers évènements internationaux (e.g. RoboCup, Exposition Universelle de Shanghai en 2010, etc.) participe à sa popularisation auprès des laboratoires de recherche, des universités et des développeurs. Une seconde génération de robot Nao apparait en 2011 (dit Nao Next Gen). L'entreprise dévoile durant la même période un nouveau projet, en partenariat avec différents acteurs de la recherche,  qui vise à créer un véritable robot d'assistance à la personne, Roméo. 

\subsection{La famille Aldebaran}
\label{Entreprise: La famille Aldebaran}
Durant l'année 2012, Aldebaran est racheté par Softbank, société spécialisée dans le commerce électronique au Japon. Débute alors la conception d'un tout nous produit, le robot humanoïde Pepper. Dévoilé au grand public en 2014, il est dans un premier temps vendu au Japon auprès des entreprises. Les premiers clients à en bénéficier seront les magasins de téléphonie mobile du groupe SoftBank. Les ventes s'ouvrent dans un second temps au particuliers Japonais. La société poursuit à présent le développement de ses trois produits afin de les améliorer et de conquérir de nouveaux marchés (en Europe , en Chine et aux États-Unis)

\section{Les produits}
\label{Entreprise: Les produits}
Aldebaran commercialise à ce jours deux produits: Nao et Pepper. Le robot Roméo est une plateforme de recherche. 

\subsection{Nao}
\label{Entreprise: Les produits: Nao}
Nao est un robot humanoïde de 58 cm de hauteur. Son public cible est principalement les laboratoires de recherche et le monde de l'éducation (allant des écoles primaires aux universités). Il est actuellement le produit phare de l'entreprise, connu auprès du grand public. 

\subsubsection{Caractéristiques techniques}
\label{Entreprise:Les produits: Nao: Caractéristiques techniques}
Caractéristiques techniques de la dernière version de Nao (V4).
\newline 

\begin{tabular}{|l|p{10cm}|}
\hline
\multicolumn{2}{|c|}{Caractéristiques générales} \\
\hline
Hauteur & 58 cm \\
\hline 
masse & 4,8 Kg \\
\hline 
Degrés de liberté  & 25 \\
\hline
Processeur & Intel Atom 1,6 GHz \\
\hline
Système d'exploitation & Middleware Aldebaran NAOqi, basé sur un noyau Linux \\
\hline 
Vision & Deux caméras frontales, 1220p, 30ips \\
\hline
Audio & 4 microphones directionnels \newline moteur de reconnaissance vocale Nuance  \\
\hline
Connectivité & Wi-Fi, Ethernet \\
\hline

\end{tabular}

développement (Chorégraphe, langages, plateformes)

\subsection{Pepper}
\label{Entreprise: Les produits: Pepper}

\subsection{Roméo}
\label{Entreprise: Les produits: Roméo}

