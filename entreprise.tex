\chapter{Entreprise}
\label{Entreprise}
\thispagestyle{fancy}

\section{Histoire}
\label{Entreprise: histoire}
Aldebaran (anciennement Aldebaran Robotics) est une société française de robotique humanoïde fondée en 2005 par Bruno MAISONNIER.

\subsection{Le premier robot, NAO}
\label{Entreprise:Histoire:Nao}
Constituée au départ d'une équipe de douze collaborateurs, la toute jeune entreprise se fixe comme objectif de développer des robots humanoïdes et de les commercialiser au grand public en tant que "nouvelle espèce bienveillante à l'égard des humains". Après trois années de recherche et développement, la société dévoile en 2008 son tout premier produit: NAO. La participation du robot humanoïde à divers évènements internationaux, comme par exemple la RoboCup ou encore l'Exposition Universelle de Shanghai en 2010 participe à sa popularisation auprès des laboratoires de recherche, des universités et des développeurs. Une seconde génération de robot NAO apparait en 2011. L'entreprise dévoile durant la même période le projet Romeo dont l'objectif est de créer un véritable robot d'assistance à la personne en partenariat avec différents acteurs de la recherche. 

\subsection{La famille s'agrandit}
\label{Entreprise: La famille Aldebaran}
Lors de l'année 2012 Aldebaran Robotics est rachetée par SoftBank, société spécialisée dans le commerce électronique au Japon et prend le nom d'Aldebaran (suppression du terme "Robotics"). Débute alors la conception d'un tout nouveau produit, le robot humanoïde Pepper. Dévoilé au grand public en 2014, il est dans un premier temps vendu au Japon auprès des entreprises. Les premiers clients à en bénéficier sont les magasins de téléphonie mobile du groupe SoftBank. Les ventes s'ouvrent par la suite aux particuliers Japonais. La société compte aujourd'hui plus de 400 collaborateurs et poursuit le développement de ses trois produits afin de les améliorer et de conquérir de nouveaux marchés (Europe , Chine et États-Unis).

\section{Les produits}
\label{Entreprise: Les produits}
Aldebaran commercialise à ce jour deux produits: NAO et Pepper. Le robot Romeo est une plateforme de recherche.

\subsection{NAO}
\label{Entreprise: Les produits: Nao}
NAO est un robot humanoïde de 58 cm de hauteur. Les publics ciblés sont essentiellement les laboratoires de recherche et le monde de l'éducation (des écoles primaires jusqu'aux universités).

\begin{figure}[H]
	\centering\includegraphics[height=7cm]{images/nao.jpg}
	\caption{Le robot humanoïde NAO}
	\label{fig:Robot humanoïde Nao}
\end{figure}

\paragraph{Caractéristiques techniques}
\label{Entreprise:Les produits: Nao: Caractéristiques techniques}
Caractéristiques techniques de la dernière version de NAO (V5, Évolution) tableau \ref{tab: Caractéristiques techniques de Nao}.

\begin{table}[H]
	\begin{tabular}{ | l | p{10cm} | }
	\hline
	\multicolumn{2}{|c|}{Caractéristiques générales} \\
	\hline
	Dimensions & 574 x 311 x 275 mm \\
	\hline 
	Masse & 5.4 kg \\
	\hline 
	Degrés de liberté  & 25 \\
	\hline
	Processeur & Intel Atom Z530 \newline 1.6 GHz \newline RAM: 1GB \newline Mémoire flash: 2GB  \newline Micro SDHC: 8 GB \\
	\hline
	Système d'exploitation & Middleware Aldebaran NAOqi basé sur un noyau Linux \\
	\hline
	Connectivité & Wi-Fi, Ethernet, USB \\
	\hline
	Batterie & Autonomie: 90 minutes en usage normal \newline Energie: 48.6 Wh \\
	\hline 
	Vision & Deux caméras frontales 2D, 1220p, 30ips \\
	\hline
	Audio & Sortie: 2 haut-parleurs stéréo \newline 4 microphones directionnels \newline moteur de reconnaissance vocale Nuance  \\
	\hline
	Capteurs & 2 capteurs infra-rouges, résistance sensible à la pression, centrale inertielle, 2 systèmes sonars, 3 surfaces tactiles \\
	\hline
	\end{tabular}
\caption[Caractéristiques techniques de NAO]{Caractéristiques techniques de la dernière version commerciale de NAO}
\label {tab: Caractéristiques techniques de Nao}
\cite{NaoTech}
\end{table}

\subsection{Pepper}
\label{Entreprise: Les produits: Pepper}
Dernier né d'Aldebaran, le robot Pepper est conçu pour vivre au côté des humains. Imaginé au départ pour accompagner et informer les clients dans les magasins de téléphonie du groupe japonais SoftBank, l'entreprise cherche à présent à placer son produit chez les particuliers. Le robot reprend la structure software et hardware de NAO. Contrairement à ce dernier, Pepper se déplace non pas grâce à une paire de jambes, mais via trois roues omnidirectionnelles qui facilitent son déplacement. A noter également que celui-ci est équipé d'une tablette tactile sur son torse pour faciliter les interactions Homme-Machine.

\begin{figure}[H]
	\centering\includegraphics[height=6cm]{images/pepper.jpg}
	\caption{Le robot humanoïde Pepper}
	\label{fig:Robot humanoïde Pepper}
\end{figure}

\paragraph{Caractéristiques techniques}
Caractéristiques techniques de la dernière version commerciale de Pepper (V1.7) tableau \ref{tab: Caractéristiques techniques de Pepper}.

\begin{table}[H]
\begin{tabular}{ | l | p{10cm} | }
	\hline
	\multicolumn{2}{|c|}{Caractéristiques générales} \\
	\hline
	Dimensions & 1210 x 480 x 425 mm \\
	\hline 
	Masse & 28 kg \\
	\hline 
	Degrés de liberté  & 20 \\
	\hline
	Processeur & Intel Atom E3845 \newline 1.91 GHz \newline RAM: 4 GB \newline Mémoire flash: 8 GB \newline MICRO SDHC: 16Go  \\
	\hline
	Système d'exploitation & Middleware Aldebaran NAOqi,\newline basé sur un noyau Linux \\
	\hline
	Connectivité & Wi-Fi, Ethernet, USB \\
	\hline
	Batterie & Énergie: 795 Wh \\
	\hline 
	Vision & 2 caméras 2D \newline 1 caméra 3D \\
	\hline
	Audio & 3 microphones directionnels \newline moteur de reconnaissance vocale Nuance  \\
	\hline
	Connectivité & Wi-Fi, Ethernet \\
	\hline
	Capteurs & 6 lasers, 2 capteurs infra-rouges, 1 système sonar, résistance sensible à la pression, 2 centrales inertielles, 3 surfaces tactiles \\
	\hline
\end{tabular}
\caption[Caractéristiques techniques de Pepper]{Caractéristiques techniques de la dernière version commerciale  de Pepper}
\label {tab: Caractéristiques techniques de Pepper}
\cite{PepperTech}
\end{table}

\subsection{Romeo}
\label{Entreprise: Les produits: Romeo}
Romeo est un nouveau type de robot d'accompagnement et d'assistance à la personne. Cette plateforme de recherche est soutenue par Aldebaran ainsi que d'autres partenaires universitaires et laboratoires de recherche (e.g. INRIA, LAAS-CNRS, ISIR, ENSTA, Telecom, etc.). Il s'agit pour l'instant d'un prototype et sert principalement de plateforme de tests pour les prochaines innovations majeures d'Aldebaran (e.g. yeux mobiles, système vestibulaire, etc.).

\begin{figure}[H]
	\centering\includegraphics[height=6cm]{images/romeo.jpg}
	\caption{Le robot humanoïde Romeo}
	\label{fig:Robot humanoïde Romeo}
\end{figure}

\paragraph{Caractéristiques techniques}
Caractéristiques techniques de la dernière version commerciale de Romeo (V2) tableau \ref{tab: Caractéristiques techniques de Romeo}.

\begin{table}[H]
	\begin{tabular}{ | l | p{10cm} | }
		\hline
		\multicolumn{2}{|c|}{Caractéristiques générales} \\
		\hline
		Hauteur & 1467 mm \\
		\hline 
		Masse & 37 kg \\
		\hline
		Processeur & Intel ATOM Z530 \newline 1.6 GHz \newline RAM: 1 GB \newline Mémoire flash: 2 GB \newline MICRO SDHC: 8 Go  \\
		\hline
		Système d'exploitation & Middleware Aldebaran NAOqi,\newline basé sur un noyau Linux \\
		\hline
		Connectivité & Wi-Fi, Ethernet \\
		\hline
		Batterie & Énergie: 795 Wh \\
		\hline 
		Vision & 4 caméras 2D \newline 1 caméra 3D \\
		\hline
		Audio & 3 microphones directionnels \newline moteur de reconnaissance vocale Nuance  \\
		\hline
		Connectivité & Wi-Fi, Ethernet \\
		\hline
		Capteurs & 6 lasers, 2 capteurs infra-rouges, 1 système sonar, résistance sensible à la pression, 2 centrales inertielles, 3 surfaces tactiles \\
		\hline
	\end{tabular}
	\caption[Caractéristiques techniques de Romeo]{Caractéristiques techniques de la dernière version de Romeo}
	\label {tab: Caractéristiques techniques de Romeo}
	\cite{RomeoTech}
\end{table}

\subsection{Le système d'exploitation NAOqi}
\label{Entreprise: Les produits: NAOqi}
NAOqi est le système d'exploitation commun aux 3 robots d'Aldebaran. Il se base sur la distribution de Linux Gentoo et contient plusieurs APIs qui permettent de commander et contrôler les robots \cite{NAOqiTech}.

\begin{description}
	\item [NAOqi Core:] Gestion de l'ensemble des fonctions de base des robots (e.g. mémoire, "Autonomous Life", comportement du robot, etc.).
	\item [NAOqi Motion:] Gestion des mouvements du robot.
	\item [NAOqi Audio:]  Gestion de la partie audio du robot.
	\item [NAOqi Vision:] Gestion de la partie vidéo du robot.
	\item [NAOqi People Perception:] Ce module est utilisé pour détecter la présence de personnes autour du robot.
	\item [NAOqi Sensors:]  Gestion de l'ensemble des capteurs qui équipent le robot.
\end{description} 


\subsection{Plateforme de développement}
\label{Entreprise:Les produits: Nao: Plateforme de développement}
Les robots sont fournis avec une plateforme de développement.
\begin{description}
	\item[Choregraphe:] Il s'agit d'un outil de programmation graphique basé sur une interface qui prend la forme de schémas blocs \cite{ChoregrapheTech}. Il permet de façon simple d'interagir avec le robot et de concevoir des applications . Il peut s'utiliser avec un environnement de simulation 3D permettant aux développeurs de tester leurs applications sans même posséder un robot. Le logiciel permet également de disposer d'un retour visuel sur ce que le robot perçoit (e.g. vidéos issues des caméras, données des moteurs, etc.)
	
	\item[Kit de développement (SDK):] Il permet de développer des applications pour les robots via plusieurs langages de programmation:  C++, Python et Java \cite{SDKTech}.
\end{description}

