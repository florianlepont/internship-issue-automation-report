\chapter{Entreprise}
\label{Entreprise}
\thispagestyle{fancy}

\section{Histoire}
\label{Entreprise: histoire}
Aldebaran (anciennement Aldebaran Robotics) est une société Française de robotique humanoïde, fondée en 2005 par Bruno Maisonnier. 

\subsection{l'Aire Nao}
\label{Entreprise:Histoire:Nao}
Constituée au départ d'une équipe de douze collaborateurs, la toute jeune entreprise se fixe comme objectif de développer des robots humanoïdes et de les commercialiser au grand public, en tant que "nouvelle espèce bienveillante à l'égard des humains" . Après trois années de recherche et développement, la société dévoile en 2008 son tout premier robot: Nao. La participation du robot humanoïde à divers évènements internationaux (e.g. RoboCup, Exposition Universelle de Shanghai en 2010, etc.) participe à sa popularisation auprès des laboratoires de recherche, des universités et des développeurs. Une seconde génération de robot Nao apparait en 2011 (dit Nao Next Gen). L'entreprise dévoile durant la même période un nouveau projet, en partenariat avec différents acteurs de la recherche,  qui vise à créer un véritable robot d'assistance à la personne, Roméo. 

\subsection{La famille Aldebaran}
\label{Entreprise: La famille Aldebaran}
Durant l'année 2012, Aldebaran est racheté par Softbank, société spécialisée dans le commerce électronique au Japon. Débute alors la conception d'un tout nous produit, le robot humanoïde Pepper. Dévoilé au grand public en 2014, il est dans un premier temps vendu au Japon auprès des entreprises. Les premiers clients à en bénéficier seront les magasins de téléphonie mobile du groupe SoftBank. Les ventes s'ouvrent dans un second temps au particuliers Japonais. La société poursuit à présent le développement de ses trois produits afin de les améliorer et de conquérir de nouveaux marchés (en Europe , en Chine et aux États-Unis)

\section{Les produits}
\label{Entreprise: Les produits}
Aldebaran commercialise à ce jours deux produits: Nao et Pepper. Le robot Roméo est une plateforme de recherche. 

\subsection{Nao}
\label{Entreprise: Les produits: Nao}
Nao est un robot humanoïde de 58 cm de hauteur. Son public cible est principalement les laboratoires de recherche et le monde de l'éducation (allant des écoles primaires aux universités). Il est actuellement le produit le plus connu de l'entreprise auprès du grand public. 

\paragraph{Caractéristiques techniques}
\label{Entreprise:Les produits: Nao: Caractéristiques techniques}
Caractéristiques techniques de la dernière version de Nao (V5). \ref{tab: Caractéristiques technique de Nao}
\newline 

\begin{table}[h]
\begin{tabular}{ | l | p{7cm} | }
\hline
\multicolumn{2}{|c|}{Caractéristiques générales} \\
\hline
Dimension & 574 x 311 x 275 mm \\
\hline 
masse & 5,4 kg \\
\hline 
Degrés de liberté  & 25 \\
\hline
Processeur & Intel Atom Z530 1.6 GHz CPU \newline 1GB RAM \newline 2GB mémoire flash \newline 8 GB Micro SDHC \\
\hline
Système d'exploitation & Middleware Aldebaran NAOqi,\newline basé sur un noyau Linux \\
\hline 
Vision & Deux caméras frontales, 1220p, 30ips \\
\hline
Audio & 4 microphones directionnels \newline moteur de reconnaissance vocale Nuance  \\
\hline
Connectivité & Wi-Fi, Ethernet \\
\hline
\end{tabular}
\caption[Caractéristiques technique de Nao]{Caractéristiques techniques de la dernière version commerciale  de Nao}
\label {tab: Caractéristiques technique de Nao}
\end{table}

\subsection{Pepper}
\label{Entreprise: Les produits: Pepper}
Dernier né d'Aldebaran, le robot Pepper est conçu pour vivre au côté des humains. Imaginé au départ pour accompagner et informer les clients dans les magasins de téléphonie du groupe japonais SoftBank, l'entreprise cherche à présent à placer son produit auprès des particuliers. Le robot se base sur la structure software et hardware de Nao. Cependant, contrairement à son compagnon Nao, celui-ci se déplace non pas grâce à une paire de jambes, mais via trois roues omnidirectionnelles, facilitant son déplacement. A noter également que Pepper est équipé d'une tablette tactile sur son torse afin de faciliter les interactions Homme-Machine.

\paragraph{Caractéristiques techniques}
Caractéristiques techniques de la dernière version commerciale de Pepper (1.6). \ref{tab: Caractéristiques technique de Pepper}

\begin{table}[h]
\begin{tabular}{ | l | p{7cm} | }
	\hline
	\multicolumn{2}{|c|}{Caractéristiques générales} \\
	\hline
	Hauteur & 121 cm \\
	\hline 
	masse & 28 kg \\
	\hline 
	Degrés de liberté  & 17 \\
	\hline
	Processeur & ?? \\
	\hline
	Système d'exploitation & Middleware Aldebaran NAOqi,\newline basé sur un noyau Linux \\
	\hline 
	Vision & 1 caméra 3D \\
	\hline
	Audio & 3 microphones directionnels \newline moteur de reconnaissance vocale Nuance  \\
	\hline
	Connectivité & Wi-Fi, Ethernet \\
	\hline
\end{tabular}
\caption[Caractéristiques technique de Pepper]{Caractéristiques techniques de la dernière version commerciale  de Pepper}
\label {tab: Caractéristiques technique de Pepper}
\end{table}

\subsection{Roméo}
\label{Entreprise: Les produits: Roméo}

\subsection{Le système d'exploitation NAOqi}
\label{Entreprise: Les produits: NAOqi}

\subsection{Plateforme de développement}
\label{Entreprise:Les produits: Nao: Plateforme de développement}
Les robots sont fournis avec une plateforme de développement.
\begin{description}
	\item[Choregraphe:] Il s'agit d'un outil de programmation graphique basé sur une interface en schéma bloc. Il permet de façon simple d'interagir avec le robot et de concevoir des applications pour le robot. Il comporte également un environnement de simulation 3D permettant au développeur de tester son application sans même posséder un robot. Le logiciel permet également de disposer d'un retour sur ce que le robot perçoit (e.g. vidéo issues des caméras, données des moteurs, etc.)
	
	\item[Kit de développement (SDK):] Permet de développer des applications pour les robots via plusieurs langages de programmation:  C++, Python et Java.
\end{description}

