\chapter{Industrialisation du produit}
\label{Industrialisation du produit}
\thispagestyle{fancy}
Une fois le processus fonctionnel de notre système défini, on l'industrialise. Cela signifie que l'on crée un ensemble d'outils permettant de l'utiliser de la manière la plus simple possible et en répondant au mieux à la problématique initiale. 

\section{Outils}
\label{Industrialisation du produit: Outils}
On soumet deux types d'outils: l'API, qui permet d'utiliser le système d'automatisation de l'investigation, et les outils graphiques, qui accompagnent l'usage de l'API et permettent à l'utilisateur d'interagir avec le programme.  

\subsection{API}
\label{Industrialisation du produit: Outils: API}
L'API que l'on propose est composée de 3 modules principaux:
\begin{description}
	\item [data base] Permet de gérer le stockage et la lecture des données nécessaires à l'exécution des algorithmes d'apprentissage. Deux types de fichiers sont sauvegardés dans la base de donnée: les fichiers logs, qui fournissent l'ensemble des exemples permettant d'entraîner l'algorithme et les fichiers générés par l'algorithme lors de son apprentissage.
	\item [data set] Gère le pré-traitement et le traitement des exemples afin de les structure pour réaliser l'apprentissage par la suite.
	\item [machine learning] Permet de créer un algorithme d'apprentissage, de l'entraîner et de l'utiliser pour investiguer sur la root cause pour laquelle il a été créé.
\end{description}

\subsubsection{Pré-traitement et traitement des données}
\label{Industrialisation du produit: Outils: API: Pré-traitement et traitement des données}
Le module "Data set" contenu dans l'API permet de traiter les exemples. 

\subsection{Outils graphiques}
\label{Industrialisation du produit: Présentation des outils: Outils graphiques}

\section{Utilisation suggérée des outils}
\label{Industrialisation du produit: Utilisation suggérée des outils}

\section{Dimensionnement de la solution}
\label{Industrialisation du produit: Dimensionnement de la solution}