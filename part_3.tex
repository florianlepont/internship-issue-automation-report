\chapter{Industrialisation du produit}
\label{Industrialisation du produit}
\thispagestyle{fancy}
Une fois le processus fonctionnel de notre système défini, on l'industrialise. Cela signifie que l'on crée un ensemble d'outils permettant de l'utiliser de la manière la plus simple possible et en répondant au mieux à la problématique initiale. 

\section{Outils}
\label{Industrialisation du produit: Outils}
On soumet deux types d'outils: l'API, qui permet d'utiliser le système d'automatisation de l'investigation, et les outils graphiques, qui accompagnent l'usage de l'API et permettent à l'utilisateur d'interagir avec le programme.  

\subsection{API}
\label{Industrialisation du produit: Outils: API}
L'API que l'on propose est composée de 3 modules principaux:
\begin{description}
	\item [data base] Permet de gérer le stockage et la lecture des données nécessaires à l'exécution des algorithmes d'apprentissage. Deux types de fichiers sont sauvegardés dans la base de donnée: les fichiers logs, qui fournissent l'ensemble des exemples permettant d'entraîner l'algorithme et les fichiers générés par l'algorithme lors de son apprentissage.
	\item [data set] Gère le pré-traitement et le traitement des exemples afin de les structurer pour réaliser l'apprentissage par la suite.
	\item [machine learning] Permet de créer un algorithme d'apprentissage, de l'entraîner et de l'utiliser pour investiguer sur la root cause pour laquelle il a été créé.
\end{description}

\subsubsection{Pré-traitement et traitement des données}
\label{Industrialisation du produit: Outils: API: Pré-traitement et traitement des données}
Le module "Data set" contenu dans l'API permet de pré-traiter les exemples et de les structurer afin de pouvoir réaliser l'apprentissage. 
Le pré-traitement est composé de 5 étapes.
\begin{description}
	\item [Lecture] La première étape du pré-traitement consiste à lire les données contenues dans le fichier log. 
	\item [Échantillonnage] Les fichiers logs générés par MEIGUI lors du déroulement du Filtering Test n'ont pas forcement tous la même période d'échantillonnage. Cela signifie que les exemples que l'on souhaite utiliser pour l'entraînement ne font pas tous la même taille. Or, les spécificités des fonctions de la librairie Scikit-learn utilisées pour l'entraînement nécessite que celles-ci aient le même nombre d'échantillons. Pour cette raison, échantillonne les données extraites des fichiers logs avec le même nombre d'échantillons. Si le nombre d'échantillons contenu dans le fichier log est inférieur au nombre d'échantillons fixé pour l'échantillonnage, on effectue un sur-échantillonnage lorsque celui-ci n'altère les données. 
	\item [Selection des motifs] Au regard de l'architecture fonctionnelle que l'on a définie en partie \ref{Automatisation du processus d'investigation: Achitecture High Level du système proposé}, on doit extraire les motifs caractéristiques d'une root cause dans chaque exemple utilisé. Pour cela, une sélection manuelle du motif est réalisée en amont via une interface graphique (c.f. \ref{Industrialisation du produit: Présentation des outils: Outils graphiques}). A partir des informations retournées par l'IHM, on est en mesure de sélectionner les données correspondant aux motifs, dans chacun de nos exemples.
	\item [Déroulement des données] On déroule ensuite les données afin de pouvoir réaliser de la reconnaissance de motifs sur plusieurs features (c.f. partie ) \ref{Automatisation du processus d'investigation: Étendre le problème à plusieurs dimensions}), i.e. que l'on place chacune des colonnes de notre matrice d'exemples l'une en dessous de l'autre pour obtenir en sortie un vecteur. 
	\item [Tri des données]
\end{description}

\subsubsection{Format des données en sortie du pré-traitement}
\label{Industrialisation du produit: Outils: API: Format des données en sortie du pré-traitement}

\subsection{Outils graphiques}
\label{Industrialisation du produit: Présentation des outils: Outils graphiques}

\section{Utilisation suggérée des outils}
\label{Industrialisation du produit: Utilisation suggérée des outils}

\section{Dimensionnement de la solution}
\label{Industrialisation du produit: Dimensionnement de la solution}