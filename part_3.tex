\chapter{Industrialisation du produit}
\label{Industrialisation du produit}
\thispagestyle{fancy}
Une fois le processus fonctionnel de notre système défini, on l'industrialise. Cela signifie que l'on crée un ensemble d'outils permettant de l'utiliser de la manière la plus simple possible et en répondant au mieux à la problématique initiale. \\
On soumet deux types d'outils: l'API, qui permet d'utiliser le système d'automatisation et les outils graphiques, qui accompagnent l'usage de l'API et offrent à l'utilisateur un moyen d'interagir avec le programme.  

\section{API}
\label{Industrialisation du produit: API}
L'API que l'on propose est composée de 3 modules :
\begin{description}
	\item [data base] Permet de gérer le stockage et la lecture des données nécessaires à l'exécution des algorithmes d'apprentissage. Deux types de fichiers y sont sauvegardés : les fichiers logs, qui fournissent l'ensemble des exemples permettant d'entraîner le SVM et les fichiers générés par l'algorithme lors de son apprentissage.
	\item [data set] Permet de prétraiter les exemples utilisés pour l'apprentissage.
	\item [machine learning] Permet de créer un algorithme d'apprentissage, de l'entraîner et de l'utiliser pour investiguer la \emph{root cause} pour laquelle il a été créé.
\end{description}

La librairie utilise le langage de programmation Python \cite{Python}. \\

\subsection{Prétraitement des données}
\label{Industrialisation du produit: API: Pré-traitement et traitement des données}
Le module "Data set" de l'API permet de prétraiter les exemples et de les structurer, afin de pouvoir entraîner le SVM. 
Le prétraitement est composé de 5 étapes.
\begin{description}
	\item [Lecture] Consiste à lire les données contenues dans le fichier log. 
	\item [Échantillonnage] Les fichiers logs générés par MEIGUI lors du déroulement du Filtering Test n'ont pas forcement tous la même période d'échantillonnage. Cela signifie que les exemples que l'on souhaite utiliser pour l'entraînement ne font pas tous la même taille. Or, les spécificités des fonctions de la librairie Scikit-learn utilisées pour l'entraînement nécessite que celles-ci aient le même nombre d'échantillons. Pour cette raison, on échantillonne les données extraites des fichiers logs ou le même nombre d'échantillons. Si le nombre d'échantillons contenus dans le fichier log est inférieur au nombre d'échantillons fixé par l'échantillonnage, on effectue un sur-échantillonnage, quand celui-ci n'altère pas les données. 
	