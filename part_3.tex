\chapter{Industrialisation du produit}
\label{Industrialisation du produit}
\thispagestyle{fancy}

\section{Définition du terme d' "industrialisation"}
\label{Industrialisation du produit: Définition du terme d' "industrialisation"}
Une fois le processus fonctionnel de notre système défini, on l'industrialise. Cela signifie que l'on crée un ensemble d'outils permettant de l'utiliser de la manière la plus simple possible et en répondant au mieux à la problématique initiale. 

\section{Présentation des outils}
\label{Industrialisation du produit: Présentation des outils}

\subsection{API}
\label{Industrialisation du produit: Présentation des outils:API}
Une API a été développer afin  

\subsection{Outils graphiques}
\label{Industrialisation du produit: Présentation des outils: Outils graphiques}

\section{Utilisation des outils suggérée}
\label{Industrialisation du produit: Mise en place du process fonctionnel}

\section{Dimensionnement de la solution}
\label{Industrialisation du produit: Utilisation des outils suggérée}

\begin{equation}
\begin{blockarray}{cccc}
& BaseAccX & BaseAccY & BaseAccZ \\
\begin{block}{c(ccc)}
Exemple_1 & log_{11} & log_{12} & log_{13} \\
Exemple_2 & log_{21} & log_{22} & log_{23} \\
Exemple_3 & log_{31} & log_{32} & log_{33} \\
... & ... & ... & ... \\
Exemple_n & log_{n1} & log_{n2} &  log_{n3} \\
\end{block}
\end{blockarray}
\end{equation}