\chapter{Industrialisation du produit}
\label{Industrialisation du produit}
\thispagestyle{fancy}
Une fois le processus fonctionnel de notre système défini, on l'industrialise. Cela signifie que l'on crée un ensemble d'outils permettant de l'utiliser de la manière la plus simple possible et en répondant au mieux à la problématique initiale. 

\section{Présentation des outils}
\label{Industrialisation du produit: Présentation des outils}
On soumet deux types d'outils: l'API, qui permet d'utiliser le système d'automatisation de l'investigation, et les outils graphiques, qui accompagnent l'usage de l'API et permettent à l'utilisateur d'interagir avec le programme.  

\subsection{API}
\label{Industrialisation du produit: Présentation des outils:API}
L'API que l'on propose est composée de 3 modules principaux:
\begin{description}
	\item [data base] Permet de gérer le stockage et la lecture des données nécéssaires 
	\item [data set]
	\item [machine learning]
\end{description}

\subsection{Outils graphiques}
\label{Industrialisation du produit: Présentation des outils: Outils graphiques}

\section{Utilisation suggérée des outils}
\label{Industrialisation du produit: Utilisation suggérée des outils}

\section{Dimensionnement de la solution}
\label{Industrialisation du produit: Dimensionnement de la solution}