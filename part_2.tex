\chapter{Utilisation du Machine Learning pour l'analyse d'incidents}
\label{Utilisation du Machine Learning pour l'analyse d'incidents}
\thispagestyle{fancy}

Au regard de l'analyse des différentes méthodes d'apprentissage automatique réalisée en partie \ref{Le Machine Learning}, on propose de tester plusieurs de ces algorithmes et de choisir celui qui répondra le plus efficacement à notre problématique initiale, i.e. d'automatiser l'analyse des erreurs apparues lors du déroulement du Filtering Test.

\section{Architecture High Level du système proposé}
\label{Utilisation du Machine Learning pour l'analyse d'incidents: Achitecture High Level du système proposé}
En s'appuyant sur les définitions et l'architecture High Level du Machine Learning (partie \ref{Le Machine Learning: Généralités sur le Machine Learning: Définition et principe général}), on propose un schéma synoptique haut niveau de la solution proposée, en réponse à la problématique. 



\section{Solutions techniques testées}
\label{Utilisation du Machine Learning pour l'analyse d'incidents: Solutions techniques testées}




\section{Solution technique proposée}
\label{Utilisation du Machine Learning pour l'analyse d'incidents: Solution technique proposée}

