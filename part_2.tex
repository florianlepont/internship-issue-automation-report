\chapter{Automatisation du processus d'investigation}
\label{Automatisation du processus d'investigation}
\thispagestyle{fancy}
Lorsqu'une \emph{error name} est révélée durant le Filtering test, de nombreuses données sont enregistrées dans un fichier journal (que l'on retrouve plus souvent sous le terme anglais de fichier "log".) Une analyse poussée de ces informations permet de déterminer la \emph{root cause} liée à l'\emph{error name} (partie \ref{Introduction:Expression du besoin:Hiérarchisation des erreurs}). Afin d'automatiser ce processus d'analyse, on s'appuie sur l'utilisation d'algorithmes d'apprentissage automatique. 

\section{Architecture High Level du système proposé}
\label{Automatisation du processus d'investigation: Achitecture High Level du système proposé}
L'architecture haut niveau de la solution que l'on propose est composée de deux couches: une couche \emph{root cause} et une couche \emph{error name}.
\begin{description}
	\item [Couche root cause] La couche \emph{root cause} permet de détecter la présence d'une \emph{root cause} dans le fichier log que l'on analyse. Il s'agit d'un algorithme d'apprentissage automatique entraîné à effectuer cette tâche.
	\item [Couche error name] La couche \emph{error name} est constituée d'un ensemble de couches \emph{root cause} de telle manière que lorsqu'un fichier log est mis en entrée du système, l'ensemble des couches \emph{root cause} sont activées. Ainsi, le système recherche la présence de chaque \emph{root cause} connue dans l'exemple étudié. On dit que les \emph{root causes} sont liées à l'\emph{error name}. On obtient en sortie de la couche \emph{error name} le nom de la \emph{root cause} ayant la plus forte probabilité d'avoir été reconnue.
\end{description} 

